\documentclass{article}

\usepackage{geometry}
\geometry{
a4paper,
top = 20mm,
bottom = 20mm,
left = 20mm,
right = 20mm}

\usepackage[portuguese]{babel}
\usepackage[utf8x]{inputenc}
\usepackage{indentfirst}
\usepackage{graphicx}
\usepackage{tikz,tikz-3dplot}
\usetikzlibrary{shapes}
\usetikzlibrary{patterns}
\usepackage[outline]{contour}
\contourlength{0.9pt}
\usepackage{wrapfig}
\usepackage{amsmath}
\usepackage{mathtools}
\usepackage{cases}
\usepackage{lipsum}
\usepackage{amsfonts}
\usepackage{amssymb}
% \usepackage{calrsfs}
% \usepackage{BOONDOX-calo}
\allowdisplaybreaks
% 
% \author{Pedroka}
% \date{Julho 2017}
\author{}
\date{}
\title{\vspace{-1cm}Soluções da lista de eletro 1, do \raisebox{\depth}{\rotatebox{180}{{\color{orange}\uppercase{\textit{\textbf{J a c k s o n}}}}}}}

\newcommand\shadetext[2][]{%
  \setbox0=\hbox{{\special{pdf:literal 7 Tr }#2}}%
  \tikz[baseline=0]\path [#1] \pgfextra{\rlap{\copy0}} (0,-\dp0) rectangle (\wd0,\ht0);%
}
\newcommand{\eval}[3]{\left. #1 \right\rvert_{#2}^{#3}}
\def\Xint#1{\mathchoice
   {\XXint\displaystyle\textstyle{#1}}%
   {\XXint\textstyle\scriptstyle{#1}}%
   {\XXint\scriptstyle\scriptscriptstyle{#1}}%
   {\XXint\scriptscriptstyle\scriptscriptstyle{#1}}%
   \!\int}
\def\XXint#1#2#3{{\setbox0=\hbox{$#1{#2#3}{\int}$}
     \vcenter{\hbox{$#2#3$}}\kern-.5\wd0}}
\def\ddashint{\Xint=}
\def\dashint{\Xint-}

\begin{document}
% \maketitle
\vspace{-1.5cm}
\section{Cilindro com bordas aterradas e potencial na superfície lateral}
\begin{wrapfigure}[13]{l}{0.4\textwidth}
\begin{tikzpicture}[>=stealth]
% \draw[pattern = north east lines] (0,0) ellipse (1.25 and 0.5);
\path[opacity=0.8,pattern color=yellow,pattern = north east lines] (0,-3.5) ellipse (1.25 and 0.5);
\draw[opacity = 0.8] (0,-3.5) -- (0,0);
\draw[->,opacity = 0.8] (0,-1.75) -- (1.8,-1.75) node[above] {$\rho$};
\draw [dashed] (-1.25,-1.75) arc (180:360:1.25 and -0.5);
\draw (-1.25,-1.75) arc (180:360:1.25 and 0.5);
\draw[opacity = 0.8] (0.625,-1.75) arc (360:290:0.625 and 0.25) node[midway,right] {\contour{white}{$\phi$}};
% \path (0.625,-1.75) arc (360:290:0.625 and 0.25) -- (0,-1.75);
\path (1,-1.75) arc (360:290:1 and 0.4) coordinate (fimarco);%-- (0,-1.75);
\draw[<-,opacity = 0.8] (fimarco) -- (0,-1.75);
\draw[fill=yellow,fill opacity=0.8] (0,0) ellipse (1.25 and 0.5);
\node at (0,0) {\contour{white}{$V(z = L) = 0$}};
\draw (-1.25,0) -- (-1.25,-3.5);
\draw (-1.25,-3.5) arc (180:360:1.25 and 0.5);
\draw [dashed] (-1.25,-3.5) arc (180:360:1.25 and -0.5);
\draw (1.25,-3.5) -- (1.25,0);
% \fill [gray,opacity=0.05] (-1.25,0) -- (-1.25,-3.5) arc (180:360:1.25 and 0.5) -- (1.25,0) arc (0:180:1.25 and -0.5);
% \path[opacity=0.8,pattern color=red,pattern = north east lines] (0,-3.5) ellipse (1.25 and 0.5);
% \path[opacity=0.5,color=red,fill] (0,-3.5) ellipse (1.25 and 0.5);
\node at (0,-3.5) {\contour{white}{$V(z=0) = 0$}};
\draw[->,opacity = 0.5] (0,0) -- (0,1) node[right] {$z$};
% \path [pattern = north west lines,opacity=0.5] (-1.25,-3.5) arc (180:360:1.25 and 0.5) -- (1.25,-3.5) arc (0:180:1.25 and -0.5);
\end{tikzpicture}
\hspace{0.2cm}
\begin{tikzpicture}
\draw (0,0) ellipse (1.25 and 0.5);
\draw (-1.25,0) -- (-1.25,-3.5);
\draw (-1.25,-3.5) arc (180:360:1.25 and 0.5);
% \draw [dashed] (-1.25,-3.5) arc (180:360:1.25 and -0.5);
\draw (1.25,-3.5) -- (1.25,0);
\path [fill,orange,opacity=0.1] (-1.25,0) -- (-1.25,-3.5) arc (180:360:1.25 and 0.5) -- (1.25,0) arc (0:180:1.25 and -0.5);
\path [pattern = north east lines,opacity=0.2] (-1.25,0) -- (-1.25,-3.5) arc (180:360:1.25 and 0.5) -- (1.25,0) arc (0:180:1.25 and -0.5);
\node at (0,-1.75) {\contour{white}{$V = V(\phi)$}};
% \path [pattern = north west lines,opacity=0.5] (-1.25,-3.5) arc (180:360:1.25 and 0.5) -- (1.25,-3.5) arc (0:180:1.25 and -0.5);
\end{tikzpicture}
\caption{Cilindro a ser estudado. Tampas aterradas e lateral com potencial dependente de $\phi$ fixo.}
\end{wrapfigure}
Vamos considerar um cilindro com as tampas aterradas e a borda submetida a um potencial $V = V(\phi)$. Nossas condições de contorno para este problema serão:
% \[
% V(\rho,\phi,z) = 
%  \begin{cases}
%   V(\rho,\phi,z = 0) &= 0\mathrm{,} \\
%   V(\rho,\phi,z = L) &= 0\mathrm{,} \\
%   V(\rho = R,\phi,z) &= V(\phi)\mathrm{.}
%  \end{cases}
% \]

\begin{subnumcases}{V(\rho,\phi,z) = }
   V(\rho,\phi,z = 0) = 0 \mathrm{,} \label{eq:zzero}\\
   V(\rho,\phi,z = L) = 0  \mathrm{,} \label{eq:zL} \\
   V(\rho = R,\phi,z) = V(\phi)\mathrm{.} \label{eq:bordas}
\end{subnumcases}

A solução da equação diferencial de Laplace em coordenadas cilíndricas, por sua vez, é dada por
\begin{multline}
 V(\rho,\phi,z) = \sum_{\nu = 0}^{\infty} \ \sum_{\mathclap{\substack{k \neq 0 \\ k = -\infty}}}^{\infty} (E_{\nu}J_{\nu}(k\rho) + F_{\nu}N_{\nu}(k \rho)) \times \\
 \times (C_{\nu}\cos{\nu \phi} + D_{\nu}\sin{\nu \phi})(A_k e^{kz} + B_k e^{-kz}) \mathrm{.}
\end{multline}

Para o potencial dentro do cilindro, nossa solução deve ser regular em todo ponto, inclusive em $\rho = 0$. A solução radial então não pode depender das funções
de Neumann, visto que estas são irregulares na origem. Podemos, ao mesmo tempo, absorver a constante ($E_{\nu}$) que acompanha a função de Bessel nas outras constantes
de nossa solução. Temos então
\begin{equation}
 V(\rho,\phi,z) = \sum_{\nu = 0}^{\infty} \ \sum_{\mathclap{\substack{k \neq 0 \\ k = -\infty}}}^{\infty} J_{\nu}(k\rho)
 (C_{\nu}\cos{\nu \phi} + D_{\nu}\sin{\nu \phi})(A_k e^{kz} + B_k e^{-kz}) \mathrm{.}
\end{equation}

Vamos utilizar a primeira condição de contorno, \eqref{eq:zzero}, em nosso potencial:
\begin{equation}
 V(\rho,\phi,0) = \sum_{\nu = 0}^{\infty} \ \sum_{\mathclap{\substack{k \neq 0 \\ k = -\infty}}}^{\infty} J_{\nu}(k\rho)
 (C_{\nu}\cos{\nu \phi} + D_{\nu}\sin{\nu \phi})(A_k + B_k) = 0 \mathrm{.}
\end{equation}

Como a condição vale para todo valor de $\rho$ e $\phi$, temos que $A_k + B_k$ deve ser nulo, $\forall k$. Desta forma, $B_k = -A_k$, e a parte longitudinal da solução
será
\begin{equation}
 A_k e^{kz} + B_k e^{-kz} = A_k (e^{kz} - e^{-kz}) = 2A_k \frac{e^{kz} - e^{-kz}}{2} \propto A_k \sinh{kz}\mathrm{,}
\end{equation}
onde podemos unificar as constantes sem problema algum.

Nosso potencial se resume agora a
\begin{equation}
 V(\rho,\phi,z) = \sum_{\nu = 0}^{\infty} \ \sum_{\mathclap{\substack{k \neq 0 \\ k = -\infty}}}^{\infty} J_{\nu}(k\rho)
 (C_{\nu}\cos{\nu \phi} + D_{\nu}\sin{\nu \phi})A_k \sinh{kz} \mathrm{.}
\end{equation}
Na segunda condição, \eqref{eq:zL}:
\begin{equation}
 V(\rho,\phi,L) = \sum_{\nu = 0}^{\infty} \ \sum_{\mathclap{\substack{k \neq 0 \\ k = -\infty}}}^{\infty} J_{\nu}(k\rho)
 (C_{\nu}\cos{\nu \phi} + D_{\nu}\sin{\nu \phi})A_k \sinh{kL} = 0\mathrm{.}
\end{equation}
Para isto ocorrer, novamente, precisamos que $\sinh{kL}$ seja nulo. Ora,
\begin{equation}
 \sinh{kL} = \frac{e^{kL} - e^{-kL}}{2} = 0 \implies e^{kL} = e^{-kL}\mathrm{,}
\end{equation}
o que só é possível caso $k$ seja puramente imaginário, visto que $L \neq 0$ (verifique se quiser). Vamos então adotar $k$ como real, e substituir $k \to i k$:
\begin{equation}
 \sinh{(ikL)} = \frac{e^{ikL} - e^{-ikL}}{2} = i \sin{kL} = 0\mathrm{.}
\end{equation}
Desta forma,
\begin{equation}
 \sin{kL} = 0 \mathrm{,}
\end{equation}
o que implica em
\begin{equation}\label{eq:ken}
 kL = n \pi,\ \forall n \in \mathbb{Z},\ \mathrm{ou,\ }k = \frac{n \pi}{L}\mathrm{.}
\end{equation}

A soma em $k$ em nossa solução para o potencial pode ser substituída por uma soma em $n$, visto que ambos se relacionam pela condição \eqref{eq:ken}. Temos assim então
\begin{equation}
 V(\rho,\phi,z) = \sum_{\nu = 0}^{\infty} \ \sum_{\mathclap{\substack{n \neq 0 \\ n = -\infty}}}^{\infty} J_{\nu}\left(i \frac{n \pi}{L}\rho \right)
 (C_{\nu}\cos{\nu \phi} + D_{\nu}\sin{\nu \phi})A_n i \sin{\left( \frac{n \pi}{L}z \right)} \mathrm{.}
\end{equation}
Se distribuirmos $A_n$ dentro do parêntesis, iremos obter duas constantes que acompanham os senos e cossenos, $A_k C_{\nu}$ e $A_k D_{\nu}$, que podem ser reescritas
como
\begin{alignat}{1}
 A_n C_{\nu} = A_{n \nu}\mathrm{,}\\
 A_n D_{\nu} = B_{n \nu}\mathrm{.}
\end{alignat}
Além de que as funções de Bessel com argumento puramente imaginário e multiplicadas de $i$ são representadas como
\begin{equation}
 i J_{\nu}\left(i \frac{n \pi}{L}\rho \right) = j_{\nu}\left( \frac{n \pi}{L}\rho \right)\mathrm{,}
\end{equation}
onde $j_{\nu}$ são denominadas as funções modificadas de Bessel.

Nosso potencial se encontra então da forma
\begin{equation}
 V(\rho,\phi,z) = \sum_{\nu = 0}^{\infty} \ \sum_{\mathclap{\substack{n \neq 0 \\ n = -\infty}}}^{\infty} j_{\nu}\left( \frac{n \pi}{L}\rho \right)
 (A_{n \nu} \cos{\nu \phi} + B_{n \nu} \sin{\nu \phi}) \sin{\left( \frac{n \pi}{L}z \right)} \mathrm{.}
\end{equation}

Vamos analisar a última condição de contorno, \eqref{eq:bordas}, sobre como fica nosso potencial nas bordas:
\begin{equation}
 V(R,\phi,z) = \sum_{\nu = 0}^{\infty} \ \sum_{\mathclap{\substack{n \neq 0 \\ n = -\infty}}}^{\infty} j_{\nu}\left( \frac{n \pi}{L} R \right)
 (A_{n \nu} \cos{\nu \phi} + B_{n \nu} \sin{\nu \phi}) \sin{\left( \frac{n \pi}{L}z \right)} = V(\phi)\mathrm{.}
\end{equation}
A fim de utilizar a ortogonalidade dos senos e cossenos, iremos multiplicar a solução do potencial por $\sin{\left( \frac{n^{\prime} \pi}{L}z \right)}$ e
$\sin{(\nu^{\prime}\phi)}$ e integrar em um período de oscilação:
\begin{multline}
 \sum_{\nu = 0}^{\infty} \ \sum_{\mathclap{\substack{n \neq 0 \\ n = -\infty}}}^{\infty}
 \int_{-L}^{L} \! \int_{-\pi}^{\pi}
 A_{n \nu} j_{\nu}\left( \frac{n \pi}{L} R \right) \cos{(\nu \phi)} \sin{\left( \frac{n \pi}{L}z \right)} \sin{(\nu^{\prime} \phi)}
 \sin{\left( \frac{n^{\prime} \pi}{L}z \right)} d\phi dz+ \\
 + \int_{-L}^{L} \!  \int_{-\pi}^{\pi}
 B_{n \nu} j_{\nu}\left( \frac{n \pi}{L} R \right) \sin{(\nu \phi)} \sin{\left( \frac{n \pi}{L}z \right)} \sin{(\nu^{\prime} \phi)}
 \sin{\left( \frac{n^{\prime} \pi}{L}z \right)} d\phi dz = \\
 = \int_{-L}^{L} \!  \int_{-\pi}^{\pi}
 V(\phi) \sin{(\nu^{\prime} \phi)} \sin{\left( \frac{n^{\prime} \pi}{L}z \right)} d\phi dz \mathrm{.}
\end{multline}

É útil lembrar das relações de integrais entre senos e cossenos num período de oscilação:
\begin{alignat}{2}
 &\int_{0}^{2 \pi} \sin{(m \theta)} \sin{(m^{\prime}\theta)} d\theta &=& \pi \delta_{m,m^{\prime}}, \mathrm{onde\ }m, m^{\prime} \geq 1 \mathrm{,}\\
 &\int_{0}^{2 \pi} \cos{(m \theta)} \sin{(m^{\prime}\theta)} d\theta &=& 0\mathrm{.}
\end{alignat}

Da segunda relação vemos de cara que a integral que acompanha $A_{n \nu}$ é zero. Logo, o que nos resta é avaliar a integral que acompanha $B_{n \nu}$.
\begin{multline}
 \int_{-L}^{L} \!  \int_{-\pi}^{\pi}
 B_{n \nu} j_{\nu}\left( \frac{n \pi}{L} R \right) \sin{(\nu \phi)} \sin{\left( \frac{n \pi}{L}z \right)} \sin{(\nu^{\prime} \phi)}
 \sin{\left( \frac{n^{\prime} \pi}{L}z \right)} d\phi dz = \\
 B_{n \nu} j_{\nu}\left( \frac{n \pi}{L} R \right)
 \int_{-\pi}^{\pi}
 \sin{(\nu \phi)}\sin{(\nu^{\prime} \phi)} d\phi \int_{-L}^{L}\sin{\left( \frac{n \pi}{L}z \right)}\sin{\left( \frac{n^{\prime} \pi}{L}z \right)}
  dz\mathrm{,}
\end{multline}
onde a primeira integra será:
\begin{equation}
 \int_{-\pi}^{\pi} \sin{(\nu \phi)}\sin{(\nu^{\prime} \phi)} d\phi = \pi \delta_{\nu,\nu^{\prime}}, \forall \nu, \nu^{\prime} \neq 0\mathrm{,}
\end{equation}
verifique se quiser. E a segunda será
\begin{equation}
 \int_{-L}^{L}\sin{\left( \frac{n \pi}{L}z \right)}\sin{\left( \frac{n^{\prime} \pi}{L}z \right)} dz =
 \frac{L}{\pi} \int_{-\pi}^{\pi}\sin{\left( n u \right) }\sin{\left( n^{\prime}u \right)} du = L \delta_{n, n^{\prime}}
\end{equation}
com a substituição $\pi z \slash L = u$. Retornando ao potencial original:
\begin{multline}
 \sum_{\nu = 0}^{\infty} \ \sum_{\mathclap{\substack{n \neq 0 \\ n = -\infty}}}^{\infty}
 \int_{-L}^{L} \!  \int_{-\pi}^{\pi}
 B_{n \nu} j_{\nu}\left( \frac{n \pi}{L} R \right) \sin{(\nu \phi)} \sin{\left( \frac{n \pi}{L}z \right)} \sin{(\nu^{\prime} \phi)}
 \sin{\left( \frac{n^{\prime} \pi}{L}z \right)} d\phi dz = \\
 = \sum_{\nu = 0}^{\infty} \ \sum_{\mathclap{\substack{n \neq 0 \\ n = -\infty}}}^{\infty} B_{n \nu} j_{\nu}\left( \frac{n \pi}{L} R \right)
 \pi \delta_{\nu,\nu^{\prime}} L \delta_{n, n^{\prime}} = B_{n^{\prime} \nu^{\prime}} j_{\nu^{\prime}}\left( \frac{n^{\prime} \pi}{L} R \right) \pi L, \forall
 \nu^{\prime} n^{\prime} \neq 0\mathrm{,}
\end{multline}
o que nos leva a
\begin{equation}
 B_{n^{\prime} \nu^{\prime}} = \frac{1}{\pi L j_{\nu^{\prime}}\left( \frac{n^{\prime} \pi}{L} R \right)}\int_{-L}^{L} \!  \int_{-\pi}^{\pi}
 V(\phi) \sin{(\nu^{\prime} \phi)} \sin{\left( \frac{n^{\prime} \pi}{L}z \right)} d\phi dz, \forall \nu^{\prime}, n^{\prime} \geq 1 \mathrm{.}
\end{equation}
Para $n^{\prime} = 0$ ou $\nu^{\prime} = 0$, uma ou outra integral será nula, então $B_{n^{\prime}0}$ e $B_{0\nu^{\prime}}$ serão nulos para
todo $n^{\prime}$ e $\nu^{\prime}$, respectivamente. Para $n^{\prime}$ negativo, a integral inverte seu sinal, dada a paridade do seno. Ou seja,
\begin{equation}
 B_{n^{\prime} \nu^{\prime}} = \frac{-1}{\pi L j_{\nu^{\prime}}\left( \frac{n^{\prime} \pi}{L} R \right)}\int_{-L}^{L} \!  \int_{-\pi}^{\pi}
 V(\phi) \sin{(\nu^{\prime} \phi)} \sin{\left( \frac{n^{\prime} \pi}{L}z \right)} d\phi dz, \forall n^{\prime} < 0, \nu^{\prime} \neq 0 \mathrm{.}
\end{equation}

Podemos multiplicar a expressão original do potencial por $\cos{(\nu^{\prime}\phi)}$ e integrar no período de oscilação, de tal forma que
\begin{multline}
 \sum_{\nu = 0}^{\infty} \ \sum_{\mathclap{\substack{n \neq 0 \\ n = -\infty}}}^{\infty}
 \int_{-L}^{L} \! \int_{-\pi}^{\pi}
 A_{n \nu} j_{\nu}\left( \frac{n \pi}{L} R \right) \cos{(\nu \phi)} \sin{\left( \frac{n \pi}{L}z \right)} \cos{(\nu^{\prime} \phi)}
 \sin{\left( \frac{n^{\prime} \pi}{L}z \right)} d\phi dz+ \\
 + \int_{-L}^{L} \!  \int_{-\pi}^{\pi}
 B_{n \nu} j_{\nu}\left( \frac{n \pi}{L} R \right) \sin{(\nu \phi)} \sin{\left( \frac{n \pi}{L}z \right)} \cos{(\nu^{\prime} \phi)}
 \sin{\left( \frac{n^{\prime} \pi}{L}z \right)} d\phi dz = \\
 = \int_{-L}^{L} \!  \int_{-\pi}^{\pi}
 V(\phi) \cos{(\nu^{\prime} \phi)} \sin{\left( \frac{n^{\prime} \pi}{L}z \right)} d\phi dz \mathrm{.}
\end{multline}
As relações integrais são análogas às apresentadas anteriormente, de tal forma que temos
\begin{alignat}{2}
 &\int_{0}^{2 \pi} \cos{(m \theta)} \cos{(m^{\prime}\theta)} d\theta &=& \pi \delta_{m,m^{\prime}}, \mathrm{onde\ }m, m^{\prime} \neq 0 \mathrm{,}\\
 &\int_{0}^{2 \pi} \cos{(m \theta)} \cos{(m^{\prime}\theta)} d\theta &=& 2\pi, \mathrm{se\ }m, m^{\prime} = 0 \mathrm{,}\\
 &\int_{0}^{2 \pi} \cos{(m \theta)} \cos{(m^{\prime}\theta)} d\theta &=& 0, \mathrm{se\ ou\ }m = 0 \mathrm{\ ou\ } m^{\prime} = 0 \mathrm{,}\\
 &\int_{0}^{2 \pi} \sin{(m \theta)} \cos{(m^{\prime}\theta)} d\theta &=& 0\mathrm{.}
\end{alignat}
De maneira análoga à anterior, para os $B_{n^{\prime} \nu^{\prime}}$, encontramos as relações para os $A_{n^{\prime} \nu^{\prime}}$ como
\begin{alignat}{1}
 A_{n^{\prime} \nu^{\prime}} &= \frac{1}{\pi L j_{\nu^{\prime}}\left( \frac{n^{\prime} \pi}{L} R \right)}\int_{-L}^{L} \!  \int_{-\pi}^{\pi}
 V(\phi) \cos{(\nu^{\prime} \phi)} \sin{\left( \frac{n^{\prime} \pi}{L}z \right)} d\phi dz, \forall \nu^{\prime}, n^{\prime} \neq 0 \mathrm{,}\\
 A_{00} &= \frac{1}{2\pi L j_{\nu^{\prime}}\left( \frac{n^{\prime} \pi}{L} R \right)}\int_{-L}^{L} \!  \int_{-\pi}^{\pi}
 V(\phi) \sin{\left( \frac{n^{\prime} \pi}{L}z \right)} d\phi dz \mathrm{,}\\
 A_{0\nu^{\prime}} &= A_{n^{\prime}0} = 0\mathrm{.}
\end{alignat}
\hfill$\blacksquare$
\section{Aplicando a solução para um potencial bipolar}
Vamos agora aplicar o resultado obtido anteriormente para encontrar o potencial causado por uma distribuição do potencial onde metade (em $\phi$) do cilindro é
mantida em $V_0$ e metade do cilindro mantida a $-V_0$.
\begin{wrapfigure}[11]{l}{0.2\textwidth}
\begin{tikzpicture}[>=stealth]
\path[pattern = north east lines] (0,0) ellipse (1.25 and 0.5);
\draw (0,0) ellipse (1.25 and 0.5);
\draw (-1.25,0) -- (-1.25,-3.5);
\draw (-1.25,-3.5) arc (180:360:1.25 and 0.5);
\path[fill,red,opacity = 0.5] (-1.25,-3.5) arc (180:270:1.25 and 0.5) -- (0,-0.5) arc (270:180:1.25 and 0.5) -- cycle;
\path[fill,red,opacity = 0.2] (-1.25,-3.5) arc (180:270:1.25 and 0.5) -- (0,0.5) arc (90:180:1.25 and 0.5) -- cycle;
% \path[fill,red,opacity = 0.5] (-1.25,-3.5) arc (180:90:1.25 and 0.5) -- (0,0.5) arc (90:180:1.25 and 0.5) -- cycle;
\path[fill,blue,opacity = 0.5] (1.25,-3.5) arc (0:-90:1.25 and 0.5) -- (0,-0.5) arc (270:360:1.25 and 0.5) -- cycle;
\path[fill,blue,opacity = 0.2] (1.25,-3.5) arc (0:-90:1.25 and 0.5) -- (0,0.5) arc (90:0:1.25 and 0.5) -- cycle;
% \path[fill,blue,opacity = 0.5] (1.25,-3.5) arc (0:90:1.25 and 0.5) -- (0,0.5) arc (90:0:1.25 and 0.5) -- cycle;
\draw [dashed] (-1.25,-3.5) arc (180:360:1.25 and -0.5);
\draw (1.25,-3.5) -- (1.25,0);
\path[pattern = north east lines,opacity=0.2] (0,-3.5) ellipse (1.25 and 0.5);
\node at (0.625,-1.75) {$-V$};
\node at (-0.625,-1.75) {$+V$};
% linha central do cilindro
% \draw (0,-0.5) -- (0,-4.0);
\end{tikzpicture}
\caption{Diagrama do potencial imposto no cilindro}
\end{wrapfigure}
As tampas ainda serão mantidas aterradas, em potencial nulo.

Podemos definir matematicamente o potencial aplicado como
\begin{equation}
V(\rho,\phi,z) = \left\{
\begin{alignedat}{3}
 &+V, \mathrm{se\ } -&\pi& &\leq \phi& < 0\\
 &-V, \mathrm{se\ } &0& &\leq \phi& < \pi
\end{alignedat}
\right.
\end{equation}

Para determinar os coeficientes da série do potencial, basta que calculemos as integrais
\begin{alignat}{1}
 &\int_{-\pi}^{\pi} V(\phi) \cos{(\nu^{\prime}\phi)} d\phi =
 \int_{-\pi}^{0} V \cos{(\nu^{\prime}\phi)} d\phi - \int_{0}^{\pi} V \cos{(\nu^{\prime}\phi)} d\phi \mathrm{,}\\
 &\int_{-\pi}^{\pi} V(\phi) \sin{(\nu^{\prime}\phi)} d\phi =
 \int_{-\pi}^{0} V \sin{(\nu^{\prime}\phi)} d\phi - \int_{0}^{\pi} V \sin{(\nu^{\prime}\phi)} d\phi \mathrm{,}\\
 &\int_{-L}^{L} \sin{\left( \frac{n^{\prime} \pi}{L}z \right)} dz \mathrm{.}
\end{alignat}

A primeira integral será ($\nu \neq 0$):
\begin{alignat}{1}
\nonumber
 \int_{-\pi}^{0} V \cos{(\nu^{\prime}\phi)} d\phi - \int_{0}^{\pi} V \cos{(\nu^{\prime}\phi)} d\phi &=
 \eval{V \frac{1}{\nu^{\prime}} \sin{(\nu^{\prime}\phi)} }{-\pi}{0} -
 \eval{V \frac{1}{\nu^{\prime}} \sin{(\nu^{\prime}\phi)} }{0}{\pi} \\ \nonumber
 &= V \frac{1}{\nu^{\prime}} \sin{(\nu^{\prime}\pi)} - V \frac{1}{\nu^{\prime}} \sin{(\nu^{\prime}\pi)} \\
 &= 0\mathrm{.}
\end{alignat}
A segunda ($\nu \neq 0$):
\begin{alignat}{1}
\nonumber
 \int_{-\pi}^{0} V \sin{(\nu^{\prime}\phi)} d\phi - \int_{0}^{\pi} V \sin{(\nu^{\prime}\phi)} d\phi &=
 \eval{-V \frac{1}{\nu^{\prime}} \cos{(\nu^{\prime}\phi)} }{-\pi}{0} +
 \eval{V \frac{1}{\nu^{\prime}} \cos{(\nu^{\prime}\phi)} }{0}{\pi} \\ \nonumber
 &= V \frac{1}{\nu^{\prime}} (\cos{(\nu^{\prime}\pi)} - 1) + V \frac{1}{\nu^{\prime}} (\cos{(\nu^{\prime}\pi)} - 1)\\
 &= \frac{2V}{\nu^{\prime}} (\cos{(\nu^{\prime}\pi)} - 1) \mathrm{.}
\end{alignat}

Logo,
\begin{alignat}{1}
 A_{n^{\prime} \nu^{\prime}} &= 0\mathrm{,}\\
 B_{n^{\prime} \nu^{\prime}} &= \frac{1}{\pi L j_{\nu^{\prime}}\left( \frac{n^{\prime} \pi}{L} R \right)}
 \left[ \frac{2V}{\nu^{\prime}} (\cos{(\nu^{\prime}\pi)} - 1) \right]
 \int_{-L}^{L} \sin{\left( \frac{n^{\prime} \pi}{L}z \right)} dz, \forall \nu^{\prime}, n^{\prime} \geq 1 \mathrm{.}
\end{alignat}

\section{Plano infinito aterrado com disco mantido a um potencial fixo}
\begin{wrapfigure}[11]{l}{0.35\textwidth}
\def\lado{2}
\def\raio{1}
\tdplotsetmaincoords{70}{110}
\begin{tikzpicture}[scale=1,tdplot_main_coords,>=stealth]
\path[pattern=north west lines] (-\lado,-\lado,0) .. controls (-\lado-1,-1) and (-\lado+1,0) ..
(-\lado,\lado,0)
-- (\lado,\lado,0) -- (\lado,-\lado,0) -- cycle;
 \path[fill,white!50!gray,opacity=0.3] (-\lado,-\lado,0) .. controls (-\lado-1,-1) and (-\lado+1,0) ..
 (-\lado,\lado,0) -- (\lado,\lado,0) -- (\lado,-\lado,0) -- cycle;
 \draw[thick,fill=white!70!blue,fill opacity = 1] (0,0,0) circle (\raio) node[left] {$V$};
 \draw[->] (0,0,0) -- (3,0,0) node[anchor=north east]{$x$};
 \draw[->] (0,0,0) -- (0,3,0) node[anchor=north west]{$y$};
 \draw[->] (0,0,0) -- (0,0,2) node[anchor=south]{$z$};
 \node[above] at (-\lado,-\lado,0) {$V = 0$};
\end{tikzpicture}
\caption{Diagrama das condições de contorno. Feio, mas ilustrativo.}
\end{wrapfigure}
Vamos considerar agora um plano infinito aterrado, onde nele há um disco circular, centrado na origem, de raio $a$, e este círculo é mantido a um potencial constante,
$V$.

As condições de contorno podem ser escritas matematicamente como
\begin{subnumcases}{V(\rho,\phi,0) = }
 V, \mathrm{\ se\ } \rho < a, \\
 0, \mathrm{\ caso\ contr\acute{a}rio}\mathrm{.}
\end{subnumcases}
A solução para a equação de Laplace em coordenadas cilíndricas é
\begin{multline}
 V(\rho,\phi,z) = \sum_{\nu = 0}^{\infty} \ \sum_{\mathclap{\substack{k \neq 0 \\ k = -\infty}}}^{\infty} (E_{\nu}J_{\nu}(k\rho) + F_{\nu}N_{\nu}(k \rho))\times\\
 \times (C_{\nu}\cos{\nu \phi} + D_{\nu}\sin{\nu \phi})(A_k e^{kz} + B_k e^{-kz}) \mathrm{.}
\end{multline}

Como nosso potencial deve desaparecer para pontos muito longe do disco, $A_k = 0$ se $k > 0$. Para $k < 0$, o mesmo acontece com $B_k$. Como a soma se extende até o
infinito, é irrelevante impormos esta condição explicitamente, sendo igualmente equivalente utilizar $A_k = 0, \forall k$, e $k>0$, apenas. Para que o potencial seja
regular na origem, $F_{\nu} = 0$. Note também que para mantermos a simetria azimutal, é necessário restringirmos nosso valor de $\nu$ para $0$ apenas. Qualquer outro
valor de $\nu$ irá causar a aparição de termos dependentes de $\phi$, tornando o potencial variante ante uma transformação $\phi \to \phi + \delta\phi$. Assim, teremos
\begin{equation}
 V(\rho,\phi,z) = \sum_{\mathclap{\substack{k \neq 0 \\ k = 0}}}^{\infty} B_k J_{0}(k\rho) e^{-kz} \mathrm{.}
\end{equation}
Como não temos restrições para um valor de $k$, vamos generalizar para o contínuo:
\begin{equation}
 V(\rho,\phi,z) = \lim_{\beta \to 0}\ \ \ \mathclap{\int_{\beta}^{\infty}}\ B(k) J_{0}(k\rho) e^{-kz} dk \mathrm{.}
\end{equation}

Sabemos que as funções de Bessel obedecem à condição
\begin{equation}
 \mathclap{\int_{0}^{\infty}}\ z J_{n}(kz)J_{n}(k^{\prime}z) dz = \frac{1}{k} \delta(k - k^{\prime})\mathrm{.}
\end{equation}
Multiplicando então a expressão para o potencial por $\rho J_{0}(k^{\prime}\rho)$ e integrando de $0$ a $\infty$ em $\rho$:
\begin{alignat}{1}
 \int_{0}^{\infty} V(\rho,\phi,z) \rho J_{0}(k^{\prime}\rho) d\rho &=
 \lim_{\beta \to 0} \int_{0}^{\infty} \ \mathclap{\int_{\beta}^{\infty}}\ B(k) J_{0}(k\rho) \rho J_{0}(k^{\prime}\rho) e^{-kz} dk d\rho \mathrm{,} \\
 &= \lim_{\beta \to 0} \ \ \ \mathclap{\int_{\beta}^{\infty}}\ B(k)
 \left( \int_{0}^{\infty} J_{0}(k\rho) \rho J_{0}(k^{\prime}\rho) d\rho \right) e^{-kz} dk \mathrm{,}\\
 &= \lim_{\beta \to 0} \ \ \ \mathclap{\int_{\beta}^{\infty}}\ B(k) \frac{1}{k}\delta(k - k^{\prime}) e^{-kz} dk \mathrm{,} \\
 &= B(k^{\prime}) \frac{1}{k^{\prime}} e^{-k^{\prime}z}, k^{\prime} > 0 \mathrm{.}
\end{alignat}
Temos então uma equação integral do tipo
\begin{equation}
 k^{\prime} e^{k^{\prime}z} \int_{0}^{\infty} V(\rho,\phi,z) \rho J_{0}(k^{\prime}\rho) d\rho = B(k^{\prime}), k^{\prime} > 0 \mathrm{.}
\end{equation}

Vamos agora considerar o plano $z = 0$ e aplicar nossas condições de contorno \textbf{(CORRIGIR ISTO, esqueci $k^{\prime}$)}:
\begin{alignat}{1}
 B(k^{\prime}) &= k^{\prime} \int_{0}^{\infty} V(\rho,\phi,0) \rho J_{0}(k^{\prime}\rho) d\rho, k^{\prime} > 0 \mathrm{,}\\
 &= k^{\prime} \int_{0}^{a} V \rho J_{0}(k^{\prime}\rho) d\rho, k^{\prime} > 0 \mathrm{,}\\
 &= k^{\prime} V \int_{0}^{a} \rho J_{0}(k^{\prime}\rho) d\rho, k^{\prime} > 0 \mathrm{,}\\
 &= k^{\prime} V \int_{0}^{a} \rho \sum_{m = 0}^{\infty}\frac{(-1)^m}{m!\Gamma(m + 1)}\left( \frac{\rho}{2} \right)^{2m} d\rho, k^{\prime} > 0 \mathrm{,}\\
 &= k^{\prime} V \sum_{m = 0}^{\infty}\frac{(-1)^m}{m!\Gamma(m + 1)}\left( \frac{1}{2} \right)^{2m} \int_{0}^{a} \rho^{2m+1} d\rho, k^{\prime} > 0 \mathrm{,}\\
 &= k^{\prime} V \sum_{m = 0}^{\infty}\frac{(-1)^m}{m!\Gamma(m + 1)}\left( \frac{1}{2} \right)^{2m} \frac{a^{2m + 2}}{2m + 2}, k^{\prime} > 0 \mathrm{,}\\
 &= k^{\prime} V \sum_{m = 0}^{\infty}\frac{(-1)^m}{m!\underbrace{(m + 1)\Gamma(m + 1)}_{=\Gamma(m+2)}}\left( \frac{1}{2} \right)^{2m+1}
 a^{2m + 2}, k^{\prime} > 0 \mathrm{,}\\
 &= k^{\prime} V a \sum_{m = 0}^{\infty}\frac{(-1)^m}{m!\Gamma(m+2)}\left( \frac{a}{2} \right)^{2m+1}, k^{\prime} > 0 \mathrm{,}\\
 &= k^{\prime} V a J_{1}(k^{\prime}a)\mathrm{.}
\end{alignat}
\shadetext[left color=magenta, right color=black!30!green, middle color=white!50!purple, shading angle=90]{\Large\bfseries F i n a l m e n t e,}
\begin{equation}
 V(\rho,\phi,z) = \lim_{\beta \to 0}\ \ \ \mathclap{\int_{\beta}^{\infty}}\ V a J_{1}(ka) J_{0}(k\rho) e^{-kz} dk \mathrm{.}
\end{equation}
\hfill$\blacksquare$

\section{Cilindro com tampa inferior a potencial fixo e tampa superior e borda aterradas}
Condições de contorno:
\begin{subnumcases}{V(\rho,\phi,z) =}
 0,\ \mathrm{se\ }\rho = R, \\
 0,\ \mathrm{se\ }z = L, \\
 V,\ \mathrm{se\ }z = 0\mathrm{.}
\end{subnumcases}

Equação de Laplace:
\begin{equation}
 V(\rho,\phi,z) = \sum_{\nu = 0}^{\infty} \ \sum_{\mathclap{\substack{k \neq 0 \\ k = -\infty}}}^{\infty} (E_{\nu}J_{\nu}(k\rho) + F_{\nu}N_{\nu}(k \rho))
 (C_{\nu}\cos{\nu \phi} + D_{\nu}\sin{\nu \phi})(A_k e^{kz} + B_k e^{-kz}) \mathrm{.}
\end{equation}

$F_{\nu} = 0$, pois queremos o potencial dentro do cilindro, e $\nu = 0$ é a única opção possível, dada a simetria azimutal. Assim:
\begin{equation}
 V(\rho,\phi,z) = \sum_{\mathclap{\substack{k \neq 0 \\ k = -\infty}}}^{\infty} J_{0}(k\rho)
 (A_{k} e^{kz} + B_{k} e^{-kz}) \mathrm{.}
\end{equation}
Condição de contorno em $z$:
\begin{equation}
 V(\rho,\phi,L) = \sum_{\mathclap{\substack{k \neq 0 \\ k = -\infty}}}^{\infty} J_{0}(k\rho)
 (A_{k} e^{kL} + B_{k} e^{-kL}) = 0 \mathrm{.}
\end{equation}
Pela ortogonalidade de $J_{0}(k^{\prime}\rho)$,
\begin{equation}
 B_k = - A_k e^{2kL}\mathrm{.}
\end{equation}
Em $z = 0$,
\begin{equation}
 V(\rho,\phi,0) = \sum_{\mathclap{\substack{k \neq 0 \\ k = -\infty}}}^{\infty} A_k J_{0}(k\rho)
 (1 - e^{2kL}) = V \mathrm{.}
\end{equation}
Multiplicando por $\rho J_0(k^{\prime}\rho)$ e integrando para aproveitar a ortogonalidade:
\begin{equation}
 \sum_{\mathclap{\substack{k \neq 0 \\ k = -\infty}}}^{\infty} \int_{0}^{\infty} A_k J_{0}(k\rho) \rho J_0(k^{\prime}\rho)
 (1 - e^{2kL}) d\rho = \int_{0}^{\infty} V \rho J_0(k^{\prime}\rho) d\rho \mathrm{.}
\end{equation}
\begin{equation}
 A_k \frac{1}{k} (1 - e^{2kL}) = V \int_{0}^{\infty} \rho J_0(k^{\prime}\rho) d\rho \mathrm{.}
\end{equation}

% 
% Vamos aplicar a condição de contorno no raio:
% \begin{equation}
%  V(R,\phi,z) = \sum_{\mathclap{\substack{k \neq 0 \\ k = -\infty}}}^{\infty} J_{0}(kR)
%  (A_{k} e^{kz} + B_{k} e^{-kz}) = 0\mathrm{.}
% \end{equation}
% Multiplicando por $J_0(k^{\prime})$
% 
% $k$, pelo visto, deve ser contínuo. Temos então que
% \begin{equation}
%  V(\rho,\phi,z) = \dashint_{-\infty}^{\infty} J_{0}(k\rho)
%  (A_{k} e^{kz} + B_{k} e^{-kz}) dk\mathrm{.}
% \end{equation}
% Vamos aplicar a condição de contorno no raio:
% % \begin{equation}
% %  V(R,\phi,z) = \sum_{\mathclap{\substack{k \neq 0 \\ k = -\infty}}}^{\infty} J_{0}(kR)
% %  (A_{k} e^{kz} + B_{k} e^{-kz}) = 0\mathrm{.}
% % \end{equation}
% \begin{equation}
%  V(R,\phi,z) = \dashint_{-\infty}^{\infty} J_{0}(kR)
%  (A_{k} e^{kz} + B_{k} e^{-kz}) dk = 0\mathrm{.}
% \end{equation}
\section{Expansão em multipolos de distribuições discretas de carga}
\begin{equation}
 q_{lm} = \int Y^{*}_{lm}(\theta^{\prime},\phi^{\prime})r^{\prime l}\rho(\vec{x}^{\prime}) d^{3}x^{\prime}\mathrm{.}
\end{equation}
A distribuição de cargas é:
\begin{equation}
 \rho(\vec{x}) = \frac{q}{a^2}\delta(r - a)\left( \delta(\phi)+\delta\left(\phi-\frac{\pi}{2}\right)-\delta(\phi-\pi)-\delta\left(\phi - \frac{3\pi}{2}\right) \right)
 \delta\left(\theta - \frac{\pi}{2}\right)\mathrm{.}
\end{equation}
Logo,
\begin{alignat}{1}
 q_{lm} &= \int Y^{*}_{lm}(\theta^{\prime},\phi^{\prime})r^{\prime l}\rho(\vec{x}^{\prime}) d^{3}x^{\prime}\mathrm{,}\\
 &= \int_{0}^{\pi} \int_{0}^{2\pi} \int_{0}^{\infty} Y^{*}_{lm}(\theta^{\prime},\phi^{\prime})r^{\prime l}\rho(\vec{x}^{\prime})
 r^{\prime 2}\sin{\theta^{\prime}} dr^{\prime} d\phi^{\prime} d\theta^{\prime} \mathrm{,}\\
 &= q a^{l} \left(Y^{*}_{lm}(\pi \slash 2,0) + Y^{*}_{lm}(\pi \slash 2,\pi \slash 2)-Y^{*}_{lm}(\pi \slash 2,\pi) -
 Y^{*}_{lm}(\pi \slash 2,3\pi \slash 2) \right)\mathrm{.}
\end{alignat}

A expansão do potencial para um dipolo, sabendo os momentos de dipolo\footnote{Substitua se quiser.}
\begin{equation}
 \Phi(\vec{x}) = \frac{1}{4 \pi \epsilon_0} \sum_{l,m}\frac{4\pi}{2l + 1} q_{lm} \frac{Y_{lm}(\theta,\phi)}{r^{l+1}}\mathrm{,}
\end{equation}
e
\begin{equation}
 Y^{*}_{lm} = (-1)^m Y_{l,-m}(\theta,\phi)\mathrm{,}
\end{equation}
caso queira usar.

Para a outra distribuição de cargas, teremos uma densidade descrita por
\begin{equation}
 \rho(\vec{x}) = q \left(-2\frac{\delta(r)}{r^2 4\pi}\underbrace{\frac{\delta(\theta)}{\sin{\theta}}}_{\mathrm{arbitr\acute{a}rio}} +
 \frac{\delta(r-a)}{a^2 2\pi} \frac{(\delta(\theta) + \delta(\theta - \pi))}{\sin{\theta}}\right)
 \underbrace{\frac{\delta(\phi)}{2\pi}}_{\mathrm{arbitr\acute{a}rio}}\mathrm{.}
\end{equation}

Vale a pena reservar um tempo para comentar a minha indignação com este método de resolução. Aqui fazemos algo que eu não gosto muito, que é tentar driblar a
``filtragem'' dos deltas para valores onde o resultado seria originalmente nulo, como no caso de $\delta(\theta)$. Uma partícula pontual situada em cima do eixo $z$
positivo terá este valor acompanhando sua densidade. Porém, em coordenadas esféricas, nosso jacobiano possui um termo multiplicativo de $\sin{\theta}$. Isto irá fazer
com que o resultado seja nulo, pois integrar $\delta(\theta)\sin{\theta}$ nos dará zero, já que $\sin{0} = 0$.

Para ``driblar'' este resultado, o que se faz é considerar $\delta(\theta)\slash \sin{\theta}$, o que, teoricamente, ``removeria'' o seno do nosso integrando,
fazendo com que o resultado da integral seja um $1$ multiplicativo, eliminando nossos problemas. Porém, isto é tão errado que eu nem sei como nomear algo do gênero,
as camadas de erro neste feito são tão grandes que chamar isto de erro chega a ser um erro absurdo com a palavra erro. O delta de dirac ``filtra'' justamente o valor
em $\theta = 0$ do integrando. Porém, neste caso, ambas as funções $\sin{\theta}$, no numerador e no denominador, serão nulas, fazendo com que a divisão de uma pela
outra seja \textbf{indefinida}. Remover o zero de nossa integração ou substituí-lo por um limite seria uma tentativa inútil, pois sem o $0$ em nossa integração, não
haveria o que o delta filtrar. Substituí-lo por um limite seria frívolo, visto que o limite da ``função'' delta de Dirac avaliada em \textbf{qualquer valor real}
é zero. A mesma só irá divergir no exato ponto determinado pelo argumento.

O delta em $\phi$ é outra arbitrariedade imensa, pois $\phi$ ali pode ser avaliado em qualquer ponto, porém, este erro não chega a ser tão crasso quanto o anterior
para $\delta(r)$ e $\delta(\theta)$.

Fechando os olhos para estas {\color{black!30!red}barbaridades}, teremos,
\begin{alignat}{1}
 q_{lm} &= \int Y^{*}_{lm}(\theta^{\prime},\phi^{\prime})r^{\prime l}\rho(\vec{x}^{\prime}) d^{3}x^{\prime}\mathrm{,}\\
 &= q a^l \left( Y^{*}_{lm}(0,0) + Y^{*}_{lm}(\pi,0) \right)\mathrm{.}
\end{alignat}

\section{Dipolo pontual como uma representação alternativa de densidade de cargas}
Vamos mostrar que a densidade de cargas
\begin{equation}\label{eq:densdipolo}
 \rho_{\mathrm{eff}}(\vec{x}) = - \vec{p}\cdot\nabla\delta(\vec{x}-\vec{x}_0)
\end{equation}
pode ser uma representação para a densidade de cargas de um dipolo.

O potencial de uma distribuição de cargas é
\begin{equation}
 \Phi(\vec{x}) = \frac{1}{4\pi \epsilon_0} \int \rho(\vec{x}^{\prime}) \frac{1}{ | \vec{x} - \vec{x}^{\prime}|} d^3 x^{\prime}\mathrm{.}
\end{equation}
Substituindo a densidade nesta expressão, teremos
\begin{alignat}{1}
\nonumber
 \Phi(\vec{x}) &= \frac{-1}{4\pi \epsilon_0} \int \vec{p}\cdot\nabla\delta(\vec{x}-\vec{x}_0) \frac{1}{ | \vec{x} - \vec{x}^{\prime}|} d^3 x^{\prime}\\ \nonumber
 &= \frac{-1}{4\pi \epsilon_0} \int \vec{p}\cdot\nabla\delta(\vec{x}-\vec{x}_0) \frac{1}{ | \vec{x} - \vec{x}^{\prime}|} d^3 x^{\prime}\\ \nonumber
 &= \frac{-1}{4\pi \epsilon_0} \vec{p}\cdot \int \nabla\delta(\vec{x}-\vec{x}_0) \frac{1}{ | \vec{x} - \vec{x}^{\prime}|} d^3 x^{\prime}\\ \nonumber
 &= \frac{1}{4\pi \epsilon_0} \vec{p}\cdot \int \delta(\vec{x}-\vec{x}_0) \nabla \frac{1}{ | \vec{x} - \vec{x}^{\prime}|} d^3 x^{\prime}
 \ \mathrm{(integrando\ por\ partes)}\\ \nonumber
 &= \frac{1}{4\pi \epsilon_0} \vec{p}\cdot \int \delta(\vec{x}-\vec{x}_0)
 \frac{(\vec{x} - \vec{x}^{\prime})}{(\vec{x} - \vec{x}^{\prime})^{3\slash 2}} d^3 x^{\prime}\\
 &= \frac{1}{4\pi \epsilon_0} \vec{p}\cdot \frac{(\vec{x} - \vec{x}_0)}{(\vec{x} - \vec{x}_0)^{3\slash 2}} =
 \frac{1}{4\pi \epsilon_0} \vec{p}\cdot \frac{(\vec{x} - \vec{x}_0)}{|\vec{x} - \vec{x}_0|^{3}}
 \mathrm{,}
\end{alignat}
que é \shadetext[left color=blue, right color=black!50!green, middle color=orange, shading angle=45]{\bfseries justamente} o potencial do dipolo.

Vamos calcular agora a energia eletrostática
\begin{alignat}{1}
 W &= \int \rho(\vec{x})\Phi(\vec{x})d^3x\\ \nonumber
 &= -\int \vec{p}\cdot\nabla\delta(\vec{x}-\vec{x}_0)\Phi(\vec{x})d^3x\ \mathrm{(integrando\ por\ partes)}\\ \nonumber
 &= \vec{p}\cdot \int \delta(\vec{x}-\vec{x}_0)\nabla\Phi(\vec{x})d^3x\\ \nonumber
 &= - \vec{p}\cdot \int \delta(\vec{x}-\vec{x}_0)\vec{E}(\vec{x})d^3x\\
 &= - \vec{p}\cdot \vec{E}(\vec{x}_0)\mathrm{,}
\end{alignat}
de acordo com o resultado esperado.

Terminamos de mostrar que, aparentemente, \eqref{eq:densdipolo} é uma distribuição de cargas de um dipolo.

\section{questão 5}
\section{Interação entre um quadrupolo e um campo elétrico}
Vamos considerar um núcleo quadrupolar, com momento de quadrupolo $Q$, submisso a um campo elétrico com simetria cilíndrica, onde o gradiente do campo é
$(\partial E \slash \partial z )_{0}$, ao longo do eixo $z$.

A energia de interação de quadrupolo é
\begin{equation}
 W = - \frac{1}{6}Q_{ij}\left( \frac{\partial E_i}{\partial x_j} \right)
\end{equation}

\section{Expansão em multipolos de distribuição contínua de cargas}
Considere a seguinte distribuição:
\begin{equation}
 \rho(\vec{r}) = \frac{1}{64\pi}r^2 e^{-r}\sin^2{\theta}\mathrm{.}
\end{equation}
Podemos substituir $\sin^2{\theta}$ por $1-\cos^2{\theta}$. $P_0(\cos{\theta}) = 1$, e $P_2(\cos{\theta}) = (1 \slash 2)(3\cos^2{\theta} - 1)$. Logo,
$\sin^2{\theta} = (2\slash 3)(P_0(\cos{\theta}) - P_2(\cos{\theta}))$. Temos então que
\begin{equation}
 \rho(\vec{r}) = \frac{1}{64\pi}r^2 e^{-r}\frac{2}{3}\left( P_0(\cos{\theta}) - P_2(\cos{\theta}) \right)\mathrm{.}
\end{equation}

Como sabemos, os momentos de dipolo são definidos por
\begin{equation}
 q_{lm} = \int Y^{*}_{lm}(\theta^{\prime},\phi^{\prime})r^{\prime l}\rho(\vec{x}^{\prime}) d^{3}x^{\prime}\mathrm{.}
\end{equation}
Teremos então
\begin{alignat}{1}
 q_{lm} &= \int Y^{*}_{lm}(\theta^{\prime},\phi^{\prime})r^{\prime l}\frac{1}{64\pi}r^{\prime 2} e^{-r}
 \frac{2}{3}\left( P_0(\cos{\theta^{\prime}}) - P_2(\cos{\theta^{\prime}}) \right) d^{3}x^{\prime} \\
 &= \int_{0}^{2 \pi} \! \int_{0}^{\pi} \! \int_{0}^{\infty} Y^{*}_{lm}(\theta^{\prime},\phi^{\prime})r^{\prime l}\frac{1}{96\pi}r^{\prime 2} e^{-r}
 \left( P_0(\cos{\theta^{\prime}}) - P_2(\cos{\theta^{\prime}}) \right) r^{\prime 2} \sin{\theta^{\prime}}dr^{\prime} d\theta^{\prime} d\phi^{\prime} \\
 &=\frac{1}{96\pi} \int_{0}^{2 \pi} \! \int_{0}^{\pi} Y^{*}_{lm}(\theta^{\prime},\phi^{\prime})\left( P_0(\cos{\theta^{\prime}}) - P_2(\cos{\theta^{\prime}}) \right)
 \int_{0}^{\infty} (r^{\prime})^{l + 4} e^{-r} dr^{\prime} \sin{\theta^{\prime}} d\theta^{\prime} d\phi^{\prime}\\
 &= \frac{1}{96\pi} \Gamma(l+5) \int_{0}^{2 \pi} \! \int_{0}^{\pi} Y^{*}_{lm}(\theta^{\prime},\phi^{\prime})
 \left( P_0(\cos{\theta^{\prime}}) - P_2(\cos{\theta^{\prime}}) \right) \sin{\theta^{\prime}} d\theta^{\prime} d\phi^{\prime}, l > -5 \mathrm{.}
\end{alignat}
Como a simetria é azimutal, teremos que $m$ deve ser zero, para termos momentos independentes de $\phi$. A relação entre os harmônicos esféricos e os polinômios de
Legendre é
\begin{alignat}{1}
 Y_{lm}(\theta,\phi) &= \sqrt{\frac{(2l + 1)}{4 \pi}\frac{(l-m)!}{(l+m)!}} P_{l}^{m}(\cos{\theta}) e^{i m \phi}\mathrm{,}\\
 Y_{l0}(\theta,\phi) &= \sqrt{\frac{(2l + 1)}{4 \pi}} P_{l}(\cos{\theta}) \mathrm{.}
\end{alignat}
Teremos então que
\begin{alignat}{1}
 q_{l0} &= \frac{1}{96\pi} \Gamma(l+5) 2\pi \int_{0}^{\pi} \sqrt{\frac{(2l + 1)}{4 \pi}} P_{l}(\cos{\theta^{\prime}})
 \left( P_0(\cos{\theta^{\prime}}) - P_2(\cos{\theta^{\prime}}) \right) \sin{\theta^{\prime}} d\theta^{\prime}, l > -5 \\
 &= \frac{1}{48} \Gamma(l+5) \sqrt{\frac{(2l + 1)}{4 \pi}}
 \left( \int_{0}^{\pi} P_{l}(\cos{\theta^{\prime}}) P_0(\cos{\theta^{\prime}})\sin{\theta^{\prime}} d\theta^{\prime} -
 \int_{0}^{\pi} P_{l}(\cos{\theta^{\prime}}) P_2(\cos{\theta^{\prime}}) \sin{\theta^{\prime}} d\theta^{\prime} \right), l > -5 \\
 &= \frac{1}{48} \Gamma(l+5) \sqrt{\frac{(2l + 1)}{4 \pi}}
 \left( \frac{2}{2l + 1}\delta_{l,0} - \frac{2}{2l + 1} \delta_{l,2} \right), l > -5 \mathrm{.}
\end{alignat}
Logo, somente dois $q_{l0}$ irão sobreviver, $q_{00}$ e $q_{20}$:
\begin{alignat}{1}
 q_{00} &= \frac{1}{24} \Gamma(5) \sqrt{\frac{1}{4 \pi}} = \sqrt{\frac{1}{4 \pi}} \mathrm{,}\\
 q_{20} &= -\frac{1}{120} \Gamma(7) \sqrt{\frac{5}{4 \pi}} = - \sqrt{\frac{45}{\pi}}\mathrm{.}
\end{alignat}

A expansão em multipolos do potencial será então
\begin{alignat}{1}
 \Phi &= \frac{1}{4 \pi \epsilon_0} \sum_{l = -\infty}^{\infty} \frac{4 \pi}{2l + 1} q_{l0} \frac{Y_{lm}(\theta,\phi)}{r^{l+1}} \\
 &= \frac{1}{4 \pi \epsilon_0} \sum_{l = -\infty}^{\infty} \sqrt{\frac{4 \pi}{2l + 1}} q_{l0} \frac{P_l(\cos{\theta})}{r^{l+1}} \\
 &= \frac{1}{4 \pi \epsilon_0} \left( \frac{1}{r} - 6 \frac{P_2(\cos{\theta})}{r^{3}} \right) \\
 &= \frac{1}{4 \pi \epsilon_0} \left( \frac{1}{r} - \frac{9\cos^2{\theta} - 3}{r^{3}} \right)\mathrm{.}
\end{alignat}
Para encontrar uma expansão do potencial próxima da origem, é necessário utilizar as funções de Green.

\section{Casca cilíndrica imersa num dielétrico com vácuo entre as cascas}

\section{Distribuições discretas de carga sugeridas pelo Zezo}
Avaliando as densidades das distribuições de carga fornecidas pelo zezo:
\begin{alignat}{1}
 \rho_a(\vec{x}) &= Q \delta(r-a)\delta\left( \theta - \frac{\pi}{2}\right)\left( -\delta(\phi) + \delta\left( \phi - \frac{\pi}{2} \right) +
 \delta\left( \phi - \pi \right) - \delta\left( \phi - \frac{3\pi}{2} \right) \right)\\
 \rho_b(\vec{x}) &= Q \delta(r-a)\delta\left( \theta - \frac{\pi}{2}\right)\left( \delta(\phi) + \delta\left( \phi - \frac{\pi}{2} \right) +
 \delta\left( \phi - \pi \right) + \delta\left( \phi - \frac{3\pi}{2} \right) \right) - \frac{4Q}{2\pi} \frac{\delta(r)}{r^2}\delta(\cos{\theta}) \\
 \rho_c(\vec{x}) &= \frac{Q}{2 \pi}\frac{\delta(r)}{r^2}\delta(\cos{\theta}) -Q\delta(r-a)\delta\left(\theta - \frac{\pi}{2}\right)
 \left( \delta(\phi) + \delta\left( \phi - \frac{\pi}{2} \right) \right)\\
 \rho_d(\vec{x}) &= Q\delta(r-a)\left( \frac{\delta(\cos{\theta}-1)}{2\pi} + \delta(\cos{\theta})\left(\delta(\phi) + \delta\left( \frac{\pi}{2} \right) \right) \right)
\end{alignat}


\section{Carga próxima a uma esfera dielétrica}
Vamos considerar uma esfera dielétrica, com uma carga pontual próxima a sua superfície. Este problema é parecido com o problema da carga próxima à esfera condutora,
onde se resolve por método das imagens. A diferença é que aqui é como se a carga imagem fosse ``ofuscada'', com apenas uma fração da imagem associada a um espelho
perfeito. Considere a carga pontual acima do eixo $z$, localizada no ponto $r = a$.

Para isto, consideremos a equação de Laplace em coordenadas esféricas com simetria azimutal
\begin{equation}
 V(r,\theta) = \sum_{l = 0}^{\infty} \left( A_l r^l + \frac{B_l}{r^{l+1}} \right) P_l (\cos{\theta})\mathrm{.}
\end{equation}
Vamos considerar primeiramente a solução para o potencial dentro da esfera. Neste caso, as constantes $B_l$ devem ser todas nulas, pois o potencial deve ser finito
na origem. Logo,
\begin{equation}
 V_{\mathrm{in}}(r,\theta) = \sum_{l = 0}^{\infty} A_l r^l P_l (\cos{\theta})\mathrm{.}
\end{equation}
Por outro lado, fora da esfera, teremos
\begin{equation}
 V_{\mathrm{out}}(r,\theta) = \frac{1}{4 \pi \epsilon_0} \frac{q}{|\vec{x} - a\hat{z}|} + V_0(r,\theta)\mathrm{,}
\end{equation}
onde o potencial de uma carga pontual se apresentará junto de um potencial causado pela presença da esfera dielétrica. Vamos dar ao potencial $V_0$ a forma de uma
solução da equação de Laplace, visto que não temos outra fonte de carga.
\begin{equation}
 V_0(r,\theta) = \sum_{l = 0}^{\infty} \frac{B_l}{r^{l+1}} P_l (\cos{\theta})\mathrm{,}
\end{equation}
pois é necessário que o potencial desapareça quando nos afastemos suficientemente da esfera e da carga.

Podemos expandir o potencial de uma carga pontual em função de dos polinômios de Legendre, de acordo uma expressão do
\raisebox{\depth}{\rotatebox{10} {{\color{orange}\bfseries \uppercase{J a c k s o n}}}},
que não lembro qual e necessito buscar o livro para olhar, perdendo o conforto do sofá. De qualquer forma,
\begin{equation}
 \frac{1}{|\vec{x} - a\hat{z}|} = \sum_{l = 0}^{\infty} \frac{r_{<}^l}{r_{>}^{l+1}} P_l(\cos{\theta})\mathrm{,}
\end{equation}
onde $r_>$ é o maior entre $r$ e $a$, e $r_<$ é o menor entre $r$ e $a$. Bom, nosso potencial geral será
\begin{alignat}{1}
 V_{\mathrm{out}}(r,\theta) &= \frac{q}{4 \pi \epsilon_0} \sum_{l = 0}^{\infty} \frac{r_{<}^l}{r_{>}^{l+1}} P_l(\cos{\theta}) +
 \sum_{l = 0}^{\infty} \frac{B_l}{r^{l+1}} P_l (\cos{\theta}) \\
 &= \frac{q}{4 \pi \epsilon_0} \left( \sum_{l = 0}^{\infty} \frac{r_{<}^l}{r_{>}^{l+1}} + \frac{B_l}{r^{l+1}} \right) P_l (\cos{\theta}) \mathrm{,}
\end{alignat}
incorporanto a constante eletrostática nos $B_l$.

Vamos avaliar as condições de contorno. Como estas se dão na superfície, teremos que, obviamente, o $r_<$ será $r$, e $r_>$, consequentemente, será $a$.

Vamos avaliar o campo elétrico tangente à superfície
\begin{alignat}{2}
 E_{\mathrm{in},\theta} &= \left. -\frac{1}{r}\frac{\partial V_{\mathrm{in}}}{\partial \theta} \right\rvert_{r = R} &=&
 \sum_{l = 0}^{\infty} A_l R^{l-1} \frac{d P_l (\cos{\theta})}{d \theta} \sin{\theta}\\
 E_{\mathrm{out},\theta} &= \left. -\frac{1}{r}\frac{\partial V_{\mathrm{out}}}{\partial \theta} \right\rvert_{r = R} &=& 
 \frac{q}{4 \pi \epsilon_0} \left( \sum_{l = 0}^{\infty} \frac{R^{l-1}}{a^{l+1}} + \frac{B_l}{R^{l+2}} \right) \frac{dP_l (\cos{\theta})}{d\theta}\sin{\theta}
\end{alignat}
Pela continuidade do campo paralelo à superfície,
\begin{alignat}{1}
 E_{\mathrm{in},\theta} &= E_{\mathrm{out},\theta} \\
 \sum_{l = 0}^{\infty} A_l R^{l-1} \frac{d P_l (\cos{\theta})}{d \theta} \sin{\theta} &=
 \frac{q}{4 \pi \epsilon_0} \left( \sum_{l = 0}^{\infty} \frac{R^{l-1}}{a^{l+1}} + \frac{B_l}{R^{l+2}} \right) \frac{dP_l (\cos{\theta})}{d\theta}\sin{\theta}\\
 A_l R^{l-1} &= \frac{q}{4 \pi \epsilon_0} \left( \frac{R^{l-1}}{a^{l+1}} + \frac{B_l}{R^{l+2}} \right) \\
 A_l &= \frac{q}{4 \pi \epsilon_0} \left( \frac{1}{a^{l+1}} + \frac{B_l}{R^{2l+1}} \right)\mathrm{.}
\end{alignat}
Avaliando agora o deslocamento elétrico perpendicular à superfície, teremos
\begin{alignat}{2}
 D_{\mathrm{in},r} &= \left. -\epsilon \frac{\partial V_{in}}{\partial r} \right\rvert_{r = R} &=&
 - \epsilon \sum_{l = 0}^{\infty} l A_l R^{l-1} P_l (\cos{\theta}) \\
 D_{\mathrm{out},r} &= \left. -\epsilon_0 \frac{\partial V_{out}}{\partial r} \right\rvert_{r = R} &=&
 -\frac{q}{4 \pi} \left( \sum_{l = 0}^{\infty} \frac{lR^{l-1}}{a^{l+1}} -(l+1) \frac{B_l}{R^{l+2}} \right) P_l (\cos{\theta})\mathrm{.}
\end{alignat}

Pela continuidade do deslocamento elétrico perpendicular à superfície, teremos
\begin{alignat}{1}
 D_{\mathrm{in},r} &= D_{\mathrm{out},r}\\
 - \epsilon \sum_{l = 0}^{\infty} l A_l R^{l-1} P_l (\cos{\theta}) &=
 -\frac{q}{4 \pi} \left( \sum_{l = 0}^{\infty} \frac{lR^{l-1}}{a^{l+1}} -(l+1) \frac{B_l}{R^{l+2}} \right) P_l (\cos{\theta}) \\
 - \epsilon l A_l R^{l-1} &= -\frac{q}{4 \pi} \left( \frac{lR^{l-1}}{a^{l+1}} -(l+1) \frac{B_l}{R^{l+2}} \right) \\
 A_l &= \frac{q}{4 \pi \epsilon} \left( \frac{1}{a^{l+1}} - \frac{(l+1)}{l} \frac{B_l}{R^{2l+1}} \right)\mathrm{.}
\end{alignat}
Já estamos com a faca e o queijo na mão, só resta resolver para $A_l$ e $B_l$:
\begin{alignat}{1}
 \frac{q}{4 \pi \epsilon_0} \left( \frac{1}{a^{l+1}} + \frac{B_l}{R^{2l+1}} \right) &=
 \frac{q}{4 \pi \epsilon} \left( \frac{1}{a^{l+1}} - \frac{(l+1)}{l} \frac{B_l}{R^{2l+1}} \right)\\
 \frac{1}{a^{l+1}} + \frac{B_l}{R^{2l+1}} &=
 \frac{\epsilon_0}{\epsilon} \left( \frac{1}{a^{l+1}} - \frac{(l+1)}{l} \frac{B_l}{R^{2l+1}} \right)\\
 \frac{B_l}{R^{2l+1}} + \frac{\epsilon_0}{\epsilon}\frac{(l+1)}{l} \frac{B_l}{R^{2l+1}}&=
 \frac{\epsilon_0}{\epsilon} \left( \frac{1}{a^{l+1}} \right) - \frac{1}{a^{l+1}}\\
 \frac{B_l}{R^{2l+1}} \left( 1 + \frac{\epsilon_0}{\epsilon}\frac{(l+1)}{l} \right) &=
 \frac{1}{a^{l+1}}\left(\frac{\epsilon_0}{\epsilon} - 1\right)\\
 B_l &= \frac{R^{2l+1}}{a^{l+1}}\frac{\left(\frac{\epsilon_0}{\epsilon} - 1\right)}{\left( 1 + \frac{\epsilon_0}{\epsilon}\frac{(l+1)}{l} \right)}\mathrm{.}
\end{alignat}
Substituindo para $A_l$, teremos que
\begin{alignat}{1}
 A_l &= \frac{q}{4 \pi \epsilon_0} \left( \frac{1}{a^{l+1}} + \frac{1}{R^{2l+1}}
 \frac{R^{2l+1}}{a^{l+1}}\frac{\left(\frac{\epsilon_0}{\epsilon} - 1\right)}{\left( 1 + \frac{\epsilon_0}{\epsilon}\frac{(l+1)}{l} \right)} \right)\\
 &= \frac{q}{4 \pi \epsilon_0} \frac{1}{a^{l+1}} \left( 1 + 
 \frac{\left(\frac{\epsilon_0}{\epsilon} - 1\right)}{\left( 1 + \frac{\epsilon_0}{\epsilon}\frac{(l+1)}{l} \right)} \right)\\
 &= \frac{q}{4 \pi \epsilon_0} \frac{1}{a^{l+1}} \left( 1 + 
 \frac{l \left(\frac{\epsilon_0}{\epsilon} - 1\right)}{\left( l + \frac{\epsilon_0}{\epsilon}(l+1) \right)} \right)\\
 &= \frac{q}{4 \pi \epsilon_0} \frac{1}{a^{l+1}} \left( \frac{\left( l + \frac{\epsilon_0}{\epsilon}(l+1) \right) + 
 l \left(\frac{\epsilon_0}{\epsilon} - 1\right)}{\left( l + \frac{\epsilon_0}{\epsilon}(l+1) \right)} \right)\\
 &= \frac{q}{4 \pi \epsilon_0} \frac{1}{a^{l+1}} \left( \frac{2\frac{\epsilon_0}{\epsilon}l+ \frac{\epsilon_0}{\epsilon}
 }{\left( l + \frac{\epsilon_0}{\epsilon}(l+1) \right)} \right)\\
 &= \frac{q}{4 \pi \epsilon} \frac{1}{a^{l+1}} \left( \frac{2l+ 1 }{\left( l + \frac{\epsilon_0}{\epsilon}(l+1) \right)} \right)\mathrm{.}\\
\end{alignat}
Desta forma, concluímos nossa expressão final para o potencial, onde temos que
\begin{equation}
 V_{\mathrm{in}}(r,\theta) = \frac{q}{4 \pi \epsilon} \frac{1}{a} \sum_{l = 0}^{\infty}
 \left( \frac{2l+ 1 }{\left( l + \frac{\epsilon_0}{\epsilon}(l+1) \right)} \right) \left( \frac{r}{a} \right)^l P_l (\cos{\theta})\mathrm{,}
\end{equation}
e
\begin{equation}
 V_{\mathrm{out}}(r,\theta) = \frac{q}{4 \pi \epsilon_0} \left( \sum_{l = 0}^{\infty} \frac{r_{<}^l}{r_{>}^{l+1}} +
 \frac{l \left(\frac{\epsilon_0}{\epsilon} - 1\right)}{\left( l + \frac{\epsilon_0}{\epsilon}(l+1) \right)}
 \frac{R^{2l+1}}{\left( ar \right)^{l+1}} \right) P_l (\cos{\theta}) \mathrm{.}
\end{equation}

Caso coloquemos $\epsilon = \epsilon_0$, ou seja, assumindo a esfera como ``vácuo'', teremos
\begin{equation}
 V_{\mathrm{in}}(r,\theta) = \frac{q}{4 \pi \epsilon_0} \frac{1}{a} \sum_{l = 0}^{\infty}
 \left( \frac{r}{a} \right)^l P_l (\cos{\theta}) = \frac{q}{4 \pi \epsilon_0} \sum_{l = 0}^{\infty} \frac{r^l}{a^{l+1}} P_l (\cos{\theta}) \mathrm{,}
\end{equation}
e
\begin{equation}
 V_{\mathrm{out}}(r,\theta) = \frac{q}{4 \pi \epsilon_0} \sum_{l = 0}^{\infty} \frac{r_{<}^l}{r_{>}^{l+1}} P_l (\cos{\theta}) \mathrm{.}
\end{equation}
Ora, mas isto é justamente o potencial de uma carga pontual no vácuo, posicionada no eixo $z$, em $z = a$, que confirma nosso resultado,
\shadetext[left color=magenta, right color=purple, middle color=orange, shading angle=45]{\bfseries \textit{``with striking fashion''}}.
\hfill $\blacksquare$
\section{Cascas esféricas meio preenchidas com dielétrico (complicado e incompleto)}
% \begin{tikzpicture}
%     \begin{scope}
%         \clip (0:2.47) arc (0:90:2.47) to[out=225,in=100,looseness=1.2] (-1.1,-1.1) to[out=-10,in=225,looseness=1.2] (0:2.47);
%         \shade[ball color=blue!30!gray!60!black,shading angle=180] (0,0) circle (2.5);
%     \end{scope}
%     \shade[ball color=green!70!gray] (0,0) circle (2);
%     \begin{scope}
%         \clip (0:1.95) arc (0:90:1.95) to[out=225,in=100] (-0.7,-0.7) to[out=-10,in=225] (0:1.95);
%         \shade[ball color=green!30!gray!60!black,shading angle=180] (0,0) circle (2);
%     \end{scope}
%     \begin{scope}
%         \clip (0:2.45) arc (0:90:2.45) to[out=225,in=100,looseness=1.2] (-1.1,-1.1) to[out=-10,in=225,looseness=1.2]
%         (0:2.45) -- (3,0) -- (3,-3) -- (-3,-3) -- (-3,3) -- (3,3) -- (3,0);
%         \shade[ball color=blue!70!gray,opacity=0.90] (0,0) circle (2.5);
%     \end{scope}
% %     \draw[stealth-,red] (0,0) -- ++(70:3) node[right] {$r=\int\limits_{0}^{2\pi}\sin(x)\ dx$};
%     \draw[stealth-,green!70!black] (-0.7,1) -- ++(135:2) node[above] {Esfera condutora};
%     \draw[stealth-,blue!90] (225:2.2) -- ++(225:1) node[below] {Esfera condutora};
% \end{tikzpicture}
% \begin{tikzpicture}
%  \draw (0,0) circle (1cm);
%  \draw (0,0) circle (2cm);
%  \draw (-2,0) -- (-1,0);
%  \draw (2,0) -- (1,0);
% \end{tikzpicture}
Vamos considerar o dielétrico no hemisfério inferior, e o vácuo no superior, de maneira a preservar a simetria azimutal. A esfera interior tem carga $Q$, e a exterior
tem sua carga igual a $-Q$.

Entre as duas esferas, teremos o potencial descrito pela equação de Laplace ordinária, representada por
\begin{equation}
 V_{\mathrm{in}}(r,\theta) = \sum_{l=0}^{\infty} \left( A_l r^l + \frac{B_l}{r^{l+1}}\right) P_l(\cos{\theta})\mathrm{.}
\end{equation}
Fora da esfera, teremos a mesma equação, porém, com os coeficientes que acompanham $r^l$ serão todos nulos, pois o potencial deverá ir a zero à medida que nos
afastemos das cargas. Ou seja,
\begin{equation}
 V_{\mathrm{out}}(r,\theta) = \sum_{l=0}^{\infty} \frac{C_l}{r^{l+1}} P_l(\cos{\theta})\mathrm{.}
\end{equation}
Vamos para as condições de contorno. Calculando o campo elétrico tangente à superfície, teremos
\begin{alignat}{2}
 E_{\mathrm{in},\theta} &= \left. - \frac{1}{r}\frac{\partial V_{\mathrm{in}}}{\partial \theta}\right\rvert_{r=b} &=&
 \sum_{l=0}^{\infty} \left( A_l b^{l-1} + \frac{B_l}{b^{l+2}}\right) \frac{d P_l(\cos{\theta})}{d\theta}\sin{\theta} \\
 E_{\mathrm{out},\theta} &= \left. - \frac{1}{r}\frac{\partial V_{\mathrm{out}}}{\partial \theta}\right\rvert_{r=b} &=&
 \sum_{l=0}^{\infty} \frac{C_l}{b^{l+2}} \frac{d P_l(\cos{\theta})}{d\theta}\sin{\theta}\mathrm{.}
\end{alignat}
Aplicando a continuidade do campo tangencial:
\begin{alignat}{1}
 E_{\mathrm{in},\theta} &= E_{\mathrm{out},\theta}\\
 \sum_{l=0}^{\infty} \left( A_l b^{l-1} + \frac{B_l}{b^{l+2}}\right) \frac{d P_l(\cos{\theta})}{d\theta}\sin{\theta} &=
 \sum_{l=0}^{\infty} \frac{C_l}{b^{l+2}} \frac{d P_l(\cos{\theta})}{d\theta}\sin{\theta} \\
 A_l b^{l-1} + \frac{B_l}{b^{l+2}} &= \frac{C_l}{b^{l+2}}\\
 A_l b^{2l+1} + B_l &= C_l\mathrm{.}
\end{alignat}
O deslocamento elétrico perpendicular à superfície será:
\begin{alignat}{2}
 D_{\mathrm{in},r} &= \left. - \epsilon_0 \frac{\partial V_{\mathrm{in}}}{\partial r} \right\rvert_{r=b} &=&
 -\epsilon_0 \sum_{l=0}^{\infty} \left( l A_l b^{l-1} - (l+1) \frac{B_l}{b^{l+2}}\right) P_l(\cos{\theta}), 0 < \theta < \frac{\pi}{2}\\
 D_{\mathrm{in},r} &= \left. - \epsilon \frac{\partial V_{\mathrm{in}}}{\partial r} \right\rvert_{r=b} &=&
 -\epsilon \sum_{l=0}^{\infty} \left( l A_l b^{l-1} - (l+1) \frac{B_l}{b^{l+2}}\right) P_l(\cos{\theta}), \frac{\pi}{2} < \theta < \pi\\
 D_{\mathrm{out},r} &= \left. - \epsilon_0 \frac{\partial V_{\mathrm{in}}}{\partial r} \right\rvert_{r=b} &=&
 \epsilon_0 \sum_{l=0}^{\infty} (l+1)\frac{C_l}{b^{l+2}} P_l(\cos{\theta})\mathrm{.}
\end{alignat}
Como temos uma carga distribuída uniformemente na superfície, $\sigma_f = Q\slash 4\pi b^2$.

{\color{red}Isso tá ficando muito complicado, melhor deixar pra lá.}

\section{Esfera dielétrica submissa a um campo uniforme}
Vamos considerar uma esfera dielétrica, e um campo uniforme na direção $z$. Temos que, para pontos muito distantes de $r = R$,
$\vec{E}(r\gg R) = E_0 \hat{z}$, e o potencial será $V(r\gg R) = - E_0 z = - E_0 r \cos{\theta}$. Assim,
\begin{equation}
 V_{\mathrm{out}}(r,\theta) = - E_0 r \cos{\theta} + \sum_{l = 0}^{\infty} \frac{B_l}{r^{l+1}} P_l(\cos{\theta})\mathrm{,}
\end{equation}
e
\begin{equation}
 V_{\mathrm{in}}(r,\theta) = \sum_{l = 0}^{\infty} A_l r^l P_l(\cos{\theta})\mathrm{.}
\end{equation}

O deslocamento elétrico perpendicular à superfície será
\begin{alignat}{2}
 D_{\mathrm{in},r} &= \left. - \epsilon \frac{\partial V_{\mathrm{in}}}{\partial r} \right\rvert_{r=b} &=&
 -\epsilon \sum_{l=0}^{\infty} l A_l b^{l-1} P_l(\cos{\theta})\\
 D_{\mathrm{out},r} &= \left. - \epsilon_0 \frac{\partial V_{\mathrm{in}}}{\partial r} \right\rvert_{r=b} &=&
 \epsilon_0 \left( E_0 \cos{\theta} + \sum_{l=0}^{\infty} (l+1)\frac{B_l}{b^{l+2}} P_l(\cos{\theta}) \right)\mathrm{.}
\end{alignat}

Como não existem cargas livres no nosso problema, o deslocamento elétrico será contínuo:
\begin{alignat}{1}
 D_{\mathrm{in},r} &= D_{\mathrm{out},r} \\
 -\epsilon \sum_{l=0}^{\infty} l A_l b^{l-1} P_l(\cos{\theta}) &= 
 \epsilon_0 \left( E_0 \cos{\theta} + \sum_{l=0}^{\infty} (l+1)\frac{B_l}{b^{l+2}} P_l(\cos{\theta}) \right)\mathrm{,}\\
\end{alignat}
que nos leva a
\begin{alignat}{1}
 -\epsilon A_1 &= \epsilon_0 E_0 + 2 \epsilon_0\frac{B_1}{b^3}\mathrm{,}\\
 -\epsilon l A_l b^{l-1} &= \epsilon_0(l+1)\frac{B_l}{b^{l+2}}\mathrm{.}
\end{alignat}
Precisamos agora encontrar uma outra relação entre os coeficientes para determiná-los.

Vamos calcular as componentes tangenciais do campo elétrico
\begin{alignat}{2}
 E_{\mathrm{in},\theta} &= \left. - \frac{1}{r}\frac{\partial V_{\mathrm{in}}}{\partial \theta}\right\rvert_{r=b} &=&
 - \sum_{l = 0}^{\infty} A_l b^{l-1} \frac{d P_l(\cos{\theta})}{d(\cos{\theta)}}\sin{\theta} \\
 E_{\mathrm{out},\theta} &= \left. - \frac{1}{r}\frac{\partial V_{\mathrm{out}}}{\partial \theta}\right\rvert_{r=b} &=&
 E_0 \sin{\theta} - \sum_{l = 0}^{\infty} \frac{B_l}{b^{l+2}} \frac{d P_l(\cos{\theta})}{d(\cos{\theta)}}\sin{\theta}\mathrm{.}
\end{alignat}

Pela continuidade da componente tangencial do campo elétrico,
\begin{alignat}{1}
 E_{\mathrm{in},\theta} &= E_{\mathrm{out},\theta} \\
 \sum_{l = 0}^{\infty} A_l b^{l-1} \frac{d P_l(\cos{\theta})}{d(\cos{\theta)}}\sin{\theta} &=
 E_0 \sin{\theta} + \sum_{l = 0}^{\infty} \frac{B_l}{b^{l+2}} \frac{d P_l(\cos{\theta})}{d(\cos{\theta)}}\sin{\theta}\mathrm{,}
\end{alignat}
nos levando a
\begin{alignat}{1}
 A_1 &= - E_0 + \frac{B_1}{b^{3}}\mathrm{,}\\
 A_l b^{l-1} &= \frac{B_l}{b^{l+2}} \implies A_l = \frac{B_l}{b^{2+1}}\mathrm{.}
\end{alignat}
Juntando as condições para $B_l$ e $A_l$ gerais, temos que
\begin{alignat}{1}
 -\epsilon l \frac{B_l}{b^{2+1}} b^{l-1} &= \epsilon_0(l+1)\frac{B_l}{b^{l+2}}\\
 B_l \left( -\epsilon l b^{l-1} - \epsilon_0(l+1) \right) &= 0, l \neq 1\mathrm{.}\\
\end{alignat}
Logo, $B_l$ deve ser zero para todo $l \neq 1$. Para $l = 1$:
\begin{alignat}{1}
 \epsilon \frac{B_1}{b^{3}} + 2 \epsilon_0 \frac{B_1}{b^3} &= \epsilon E_0 - \epsilon_0 E_0 \mathrm{,}\\
 B_1 &= b^3 E_0 \frac{\epsilon - \epsilon_0}{\epsilon + 2\epsilon_0} \\
 A_1 &= - E_0 + \frac{b^3 E_0 \frac{\epsilon - \epsilon_0}{\epsilon + 2\epsilon_0}}{b^{3}}\\
 &= - E_0 + E_0 \frac{\epsilon - \epsilon_0}{\epsilon + 2\epsilon_0}\\
 &= E_0 \left( \frac{\epsilon - \epsilon_0 - \epsilon - 2\epsilon_0}{\epsilon + 2\epsilon_0} \right)\\
 &= E_0 \left( \frac{-3\epsilon_0}{\epsilon + 2\epsilon_0} \right)\mathrm{.}
\end{alignat}

O potencial dentro da esfera será
\begin{equation}
 V_{\mathrm{in}}(r,\theta) = - E_0 \left( \frac{3\epsilon_0}{\epsilon + 2\epsilon_0} \right) r\cos{\theta} =
 - E_0 \left( \frac{3\epsilon_0}{\epsilon + 2\epsilon_0} \right) z\mathrm{,}
\end{equation}
e o campo será
\begin{equation}
 E_{\mathrm{in}} = E_0 \left( \frac{3\epsilon_0}{\epsilon + 2\epsilon_0} \right)\mathrm{.}
\end{equation}

Teremos então um campo uniforme dentro da esfera.

\section{Esfera dielétrica com polarização uniforme}
Vamos considerar agora uma esfera dielétrica, apresentando uma polarização uniforme. Por questões de preservação da simetria azimutal, consideremos a polarização
paralela ao eixo $z$, ou seja, $\vec{P} = P_0 \hat{z}$.

Comecemos escrevendo as densidades volumétrica e superficial de cargas ligadas:
\begin{alignat}{1}
 \rho_b &= -\nabla \cdot \vec{P} = 0\mathrm{,}\\
 \sigma_b &= \vec{P}\cdot \hat{n} = P \cos{\theta}\mathrm{.}
\end{alignat}
Ora, o que nos resta agora é encontrar o potencial produzido por uma densidade superficial de cargas análoga, numa esfera de raio $R$.

A solução da equação de Laplace nos dá
\begin{equation}
 V_{\mathrm{in}} = \sum_{}
\end{equation}

\begin{equation}
 \frac{1}{4 \pi \epsilon_0} \int \frac{\sigma_b}{\mathcal{r}} da^{\prime}
\end{equation}


\section{Esfera dielétrica com carga pontual na origem}

\section{Esfera dielétrica com dipolo simples na origem}

\end{document}
