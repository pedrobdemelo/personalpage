\documentclass{article}

\usepackage[portuguese]{babel}
\usepackage[utf8x]{inputenc}
\usepackage{indentfirst}
\usepackage{geometry}
\geometry{
a4paper,
top=20mm,
bottom=20mm,
left=20mm,
right=20mm
}
\usepackage{lipsum}
\usepackage{amsmath}
\usepackage{mathtools}
\newcommand{\reais}{{\rm I\!R}}

\begin{document}
\section{Prova das propriedades do delta de Dirac}
\subsection{Avaliar o $\delta(\alpha (x-x_0))$}
Vamos considerar a integral
\begin{equation}
 \int_{-\infty}^{\infty} f(x)\delta(\alpha (x-x_0)) dx.
\end{equation}
Substituindo $u = \alpha (x-x_0)$, teremos que $du = \alpha dx$. No caso em que $\alpha > 0$, teremos que a integral irá naturalmente de um valor negativo para um
positivo, mas caso $\alpha$ seja menor que zero, a integral irá de um valor positivo para um negativo, invertendo o sentido, onde teremos que alterar o sinal da integral
para o resultado ser correto. Logo, separaremos em dois casos:
\begin{alignat}{1}
 &\int_{-\infty}^{\infty} f\left(\frac{u}{\alpha} + x_0\right)\delta(u) \frac{du}{\alpha},\ \alpha > 0,\\
 -&\int_{-\infty}^{\infty} f\left(\frac{u}{\alpha} + x_0\right)\delta(u) \frac{du}{\alpha},\ \alpha < 0.
\end{alignat}

Estas expressões podem ser unidas em uma caso consideremos o módulo de $\alpha$ e reconheçamos que $\alpha = |\alpha|$ no primeiro caso e $\alpha = -|\alpha|$ no
segundo. Assim,
\begin{equation}
 \int_{-\infty}^{\infty} f\left(\frac{u}{\alpha} + x_0\right)\delta(u) \frac{du}{|\alpha|}.
\end{equation}

Pela definição de $\delta{(u)}$:
\begin{equation}\label{eq:fxzerosobrealfa}
 \int_{-\infty}^{\infty} f\left(\frac{u}{\alpha} + x_0\right)\delta(u) \frac{du}{|\alpha|} = \frac{f(x_0)}{|\alpha|}.
\end{equation}

Finalmente,
\begin{equation}\label{eq:deltadealfa}
 \int_{-\infty}^{\infty} f(x)\delta(\alpha (x-x_0)) dx = \frac{f(x_0)}{|\alpha|}.
\end{equation}

\subsection{Avaliar o $\delta^{\prime}(x)$}
Vamos considerar a integral
\begin{equation}
 \int_{-\infty}^{\infty} f(x)\frac{d\delta(x)}{dx} dx.
\end{equation}

Integrando por partes, teremos que
\begin{equation}
 \int_{-\infty}^{\infty} f(x)\frac{d\delta(x)}{dx} dx = \left. f(x)\delta(x)\right|_{-\infty}^{\infty} - \int_{-\infty}^{\infty} \frac{df(x)}{dx}\delta(x) dx.
\end{equation}
Mas $f(x)\delta(x) = 0,\ \forall x \neq 0$, pela própria definição do delta. Logo,
\begin{equation}
 \int_{-\infty}^{\infty} f(x)\frac{d\delta(x)}{dx} dx = - \int_{-\infty}^{\infty} \frac{df(x)}{dx}\delta(x) dx.
\end{equation}

Concluímos então que
\begin{equation}
 \int_{-\infty}^{\infty} f(x)\frac{d\delta(x)}{dx} dx = - \frac{df(0)}{dx}.
\end{equation}

\subsection{Avaliar o $\delta((x-x_1)(x-x_2))$}
Vamos considerar a integral
\begin{equation}\label{eq:deltax1x2}
 \int_{-\infty}^{\infty} f(x)\delta((x-x_1)(x-x_2)) dx.
\end{equation}

Pelas propriedades da função delta, a integral $\int f(x)\delta(x)dx$ é nula se integrada em um intervalo que não contém o $0$, e igual a $f(0)$ caso contrário. Isto pode
ser generalizado ``trocando'' $x$ por uma função real contínua em suas raízes. Assim, a integral (ex $\int f(x)\delta(g(x))dx$) será nula caso integrada em um intervalo
que não contém nenhuma raiz do argumento (ex $g(x)$) da delta (ex $\delta(g(x))$), pois $\delta{(\xi)} = 0,\ \forall\ \xi \in \reais,\ \xi \neq 0$.

Desta forma, teremos que a integral \eqref{eq:deltax1x2} poderá ser decomposta em
\begin{alignat}{1}
\nonumber
 \int_{-\infty}^{\infty} f(x)\delta((x-x_1)(x-x_2)) dx = \int_{x_1 - \epsilon_1}^{x_1 + \epsilon_1} &f(x)\delta((x-x_1)(x-x_2)) dx +\\
 &+ \int_{x_2 - \epsilon_2}^{x_2 + \epsilon_2} f(x)\delta((x-x_1)(x-x_2)) dx,\label{eq:somadeltapartes}
\end{alignat}
onde $\epsilon_1$ é escolhido de maneira a garantir tanto a continuidade de $f$ em $x_1$ quanto a existência de apenas esta raiz no intervalo de integração. A mesma coisa
pode ser dita para $\epsilon_2$, $f$, $x_2$.

Vamos agora considerar $\chi$ como sendo o conjunto $(x_1 - \epsilon_1,x_1 + \epsilon_1)$, e $y \in \chi$. Relembrando a propriedade \eqref{eq:deltadealfa}, fica
evidente que
\begin{alignat}{1}
 \int_{x_1 - \epsilon_1}^{x_1 + \epsilon_1} f(x)\delta((x-x_1)(x-x_2)) dx &\leq
 \lim \sup_{y \in \chi} \int_{x_1 - \epsilon_1}^{x_1 + \epsilon_1} f(x)\delta((x-x_1)y) dx\\
 &= \int_{x_1 - \epsilon_1}^{x_1 + \epsilon_1} f(x)\delta((x-x_1)(x_1 - \epsilon_1 - x_2)) dx\\
 &= \frac{1}{|x_1 - \epsilon_1 - x_2|}\int_{x_1 - \epsilon_1}^{x_1 + \epsilon_1} f(x)\delta(x-x_1) dx.
\end{alignat}
De maneira análoga para o \textit{infimum},
\begin{alignat}{1}
 \int_{x_1 - \epsilon_1}^{x_1 + \epsilon_1} f(x)\delta((x-x_1)(x-x_2)) dx &\geq
 \lim \inf_{y \in \chi} \int_{x_1 - \epsilon_1}^{x_1 + \epsilon_1} f(x)\delta((x-x_1)y) dx\\
 &= \int_{x_1 - \epsilon_1}^{x_1 + \epsilon_1} f(x)\delta((x-x_1)(x_1 + \epsilon_1 - x_2)) dx\\
 &= \frac{1}{|x_1 + \epsilon_1 - x_2|}\int_{x_1 - \epsilon_1}^{x_1 + \epsilon_1} f(x)\delta(x-x_1) dx.
\end{alignat}

Logo, chegamos à conclusão de que
\begin{alignat}{2}
 \lim \inf_{y \in \chi} \int_{x_1 - \epsilon_1}^{x_1 + \epsilon_1} f(x)\delta((x-x_1)y) dx \leq \int_{x_1 - \epsilon_1}^{x_1 + \epsilon_1} f(x)\delta((x-x_1)(x-x_2))
 \leq \lim \sup_{y \in \chi} \int_{x_1 - \epsilon_1}^{x_1 + \epsilon_1} f(x)\delta((x-x_1)y) dx,\\
 \frac{1}{|x_1 + \epsilon_1 - x_2|}\int_{x_1 - \epsilon_1}^{x_1 + \epsilon_1} f(x)\delta(x-x_1) dx \leq \int_{x_1 - \epsilon_1}^{x_1 + \epsilon_1} f(x)\delta((x-x_1)(x-x_2))
 \leq \frac{1}{|x_1 - \epsilon_1 - x_2|}\int_{x_1 - \epsilon_1}^{x_1 + \epsilon_1} f(x)\delta(x-x_1) dx.
\end{alignat}

Consequentemente, existe um $\epsilon_1$ tal que
\begin{equation}
 \frac{1}{|x_1 - x_2|}\int_{x_1 - \epsilon_1}^{x_1 + \epsilon_1} f(x)\delta(x-x_1) dx \leq \int_{x_1 - \epsilon_1}^{x_1 + \epsilon_1} f(x)\delta((x-x_1)(x-x_2))
 \leq \frac{1}{|x_1 - x_2|}\int_{x_1 - \epsilon_1}^{x_1 + \epsilon_1} f(x)\delta(x-x_1) dx.
\end{equation}
Pelo teorema do confronto,
\begin{equation}
 \int_{x_1 - \epsilon_1}^{x_1 + \epsilon_1} f(x)\delta((x-x_1)(x-x_2)) = \frac{1}{|x_1 - x_2|}\int_{x_1 - \epsilon_1}^{x_1 + \epsilon_1} f(x)\delta(x-x_1) dx.
\end{equation}

Procedendo de maneira análoga para $x_2$, teremos que
\begin{equation}
 \int_{x_2 - \epsilon_2}^{x_2 + \epsilon_2} f(x)\delta((x-x_1)(x-x_2)) = \frac{1}{|x_2 - x_1|}\int_{x_2 - \epsilon_2}^{x_2 + \epsilon_2} f(x)\delta(x-x_2) dx.
\end{equation}

Retormando \eqref{eq:somadeltapartes}, concluímos que
\begin{alignat}{1}
 \int_{-\infty}^{\infty} f(x)\delta((x-x_1)(x-x_2)) dx &= \frac{1}{|x_1 - x_2|}\int_{x_1 - \epsilon_1}^{x_1 + \epsilon_1} f(x)\delta(x-x_1) dx + 
 \frac{1}{|x_2 - x_1|}\int_{x_2 - \epsilon_2}^{x_2 + \epsilon_2} f(x)\delta(x-x_2) dx\\
 &= \frac{1}{|x_1 - x_2|}\int_{x_1 - \epsilon_1}^{x_1 + \epsilon_1} f(x)\delta(x-x_1) dx - 
 \frac{1}{|x_1 - x_2|}\int_{x_2 - \epsilon_2}^{x_2 + \epsilon_2} f(x)\delta(x-x_2) dx\\
 &= \frac{1}{|x_1 - x_2|}\left( \int_{x_1 - \epsilon_1}^{x_1 + \epsilon_1} f(x)\delta(x-x_1) dx - \int_{x_2 - \epsilon_2}^{x_2 + \epsilon_2} f(x)\delta(x-x_2) dx \right).
\end{alignat}
Resgatando os limites originais de integração, devido às propriedades do delta:
\begin{alignat}{1}
 \int_{-\infty}^{\infty} f(x)\delta((x-x_1)(x-x_2)) dx 
 &= \frac{1}{|x_1 - x_2|}\left( \int_{x_1 - \epsilon_1}^{x_1 + \epsilon_1} f(x)\delta(x-x_1) dx - \int_{x_2 - \epsilon_2}^{x_2 + \epsilon_2} f(x)\delta(x-x_2) dx \right)\\
 &= \frac{1}{|x_1 - x_2|}\left( \int_{-\infty}^{\infty} f(x)\delta(x-x_1) dx - \int_{-\infty}^{\infty} f(x)\delta(x-x_2) dx \right)\\
 &= \frac{1}{|x_1 - x_2|}\int_{-\infty}^{\infty} f(x)\left( \delta(x-x_1) - \delta(x-x_2) \right) dx.\\
\end{alignat}

Finalmente,
\begin{equation}
 \delta((x-x_1)(x-x_2)) = \frac{\delta(x-x_1) - \delta(x-x_2)}{|x_1 - x_2|}.
\end{equation}

\subsection{Avaliar o $\delta(f)$}
Como $f(x_0) = 0$, vamos expandir $f(x)$ em torno deste ponto:
\begin{equation}
 f(x) = f(x_0) + (x - x_0)f^{\prime}(x_0) + \ldots.
\end{equation}
Porém, $f(x_0) = 0$. Logo,
\begin{equation}
 f(x) = (x - x_0)f^{\prime}(x_0) + \ldots.
\end{equation}

Então, o $\delta(f(x))$ será
\begin{equation}\label{eq:expansao}
 \delta(f(x)) = \delta((x - x_0)f^{\prime}(x_0) + \ldots).
\end{equation}

Integrando em todo o espaço, teremos
\begin{equation}
 \int_{-\infty}^{\infty}g(x)\delta(f(x))dx = \int_{x_0 - \epsilon}^{x_0 + \epsilon}g(x)\delta(f(x))dx,
\end{equation}
pois $f(x)$ só é nula em $x = x_0$. Todos os outros valores são irrelevantes para a delta.

Substituindo a expansão \eqref{eq:expansao}:
\begin{equation}
 \int_{x_0 - \epsilon}^{x_0 + \epsilon}g(x)\delta(f(x))dx = \int_{x_0 - \epsilon}^{x_0 + \epsilon} g(x)\delta((x - x_0)f^{\prime}(x_0) + \ldots) dx.
\end{equation}

Tomando um limite de $\epsilon$ tão pequeno quanto desejado, os termos de ordem $n=2$ ou superiores em $x^n$ podem ser desprezados. Logo,
\begin{equation}
 \int_{x_0 - \epsilon}^{x_0 + \epsilon} g(x)\delta((x - x_0)f^{\prime}(x_0) + \ldots) dx = \int_{x_0 - \epsilon}^{x_0 + \epsilon} g(x)\delta((x - x_0)f^{\prime}(x_0)) dx.
\end{equation}
Utilizando as propriedades dos quesitos anteriores para a função Delta:
\newline($\int \delta(kx)g(x)dx = |k|^{-1}g(0) = |k|^{-1}\int \delta(x)g(x)dx$ e $\int \delta(x-x_0)g(x)dx = g(x_0)$)
\begin{equation}
 \int_{x_0 - \epsilon}^{x_0 + \epsilon} g(x)\delta((x - x_0)f^{\prime}(x_0)) dx =
 \frac{1}{|f^{\prime}(x_0)|}\int_{x_0 - \epsilon}^{x_0 + \epsilon} g(x)\delta((x - x_0)) dx.
\end{equation}
Temos então que
\begin{alignat}{1}
 \int_{x_0 - \epsilon}^{x_0 + \epsilon}g(x)\delta(f(x))dx &= \int_{x_0 - \epsilon}^{x_0 + \epsilon} g(x)\delta((x - x_0)f^{\prime}(x_0)) dx,\\
 &= \frac{1}{|f^{\prime}(x_0)|}\int_{x_0 - \epsilon}^{x_0 + \epsilon} g(x)\delta(x-x_0)dx,\\
 &= \int_{x_0 - \epsilon}^{x_0 + \epsilon} \frac{1}{|f^{\prime}(x_0)|} g(x)\delta(x-x_0)dx.
\end{alignat}

Como a igualdade
\begin{equation}
 \int_{x_0 - \epsilon}^{x_0 + \epsilon}g(x)\delta(f(x))dx = \int_{x_0 - \epsilon}^{x_0 + \epsilon} \frac{1}{|f^{\prime}(x_0)|} g(x)\delta(x-x_0)dx
\end{equation}
vale para qualquer $g(x)$ contínua em $x = x_0$, os integrantes devem ser iguais:
\begin{equation}
 g(x)\delta(f(x)) = \frac{1}{|f^{\prime}(x_0)|} g(x)\delta(x-x_0).
\end{equation}

Consequentemente,
\begin{equation}
 \delta(f(x)) = \frac{1}{|f^{\prime}(x_0)|} \delta(x-x_0),
\end{equation}
independente do valor de $g(x)$.
\end{document}
